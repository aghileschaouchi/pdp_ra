\documentclass[12pt]{article}
\usepackage[hyphens,spaces,obeyspaces]{url}
\usepackage[colorlinks,allcolors=blue]{hyperref}
\usepackage[numbers]{natbib}
\bibliographystyle{plainnat} 

\begin{document}
\section{Elements bibliographiques}

\begin{itemize}
  \item Pour bien comprendre le concept et l'application concrète de la réalité augmentée, nous avons choisi le livre "UNDERSTANDING AUGMENTED REALITY "\cite{Ref1} qui explique le fonctionnement de la réalité augmentée, et ses différentes applications dans divers domaines.
  \item Afin d'apprendre plus sur les cas d'utilisation de la technologie rasberry Pi, on a eu recours à cet article qui nous a éclairés sur l'utilisation de rasberry pi afin de sécuriser un domicile ainsi que sur la manière d'automatiser le fonctionnement des appareils \'electrom\'enagers \cite{Ref2}.
  \item Afin de tester nos travaux, nous devrons avoir recours à un nano-ordinateur très certainement du type Raspberrypi, pour cela, nous avons inclus la documentation en ligne et officielle de la technologie Raspberrypi \cite{Ref3}.
  \item Nous pouvons aussi utiliser une technologie Arduino à fin de mettre en oeuvre notre travail, pour cela, nous avons inclus la documentation en ligne et officielle de la technologie Arduino \cite{Ref4}. 

\bibliography{pdp_bib.bib} %ref au fichier pdp_bib.bib
\end{itemize}

\end{document}



\documentclass[12pt,a4paper]{article}

\usepackage{fullpage}
\usepackage[utf8]{inputenc}
\usepackage{amsfonts}
\usepackage{amssymb}
\usepackage{listings}
\usepackage[french]{babel}
\usepackage[cyr]{aeguill}
\usepackage[hyphens,spaces,obeyspaces]{url}
\usepackage[colorlinks,allcolors=blue]{hyperref}
\usepackage[numbers]{natbib}
\bibliographystyle{plainnat} 

\def\dollar{\$}

\setlength{\parindent}{2em}
\setlength{\parskip}{1em}

\newcommand{\quotes}[1]{``#1''}

\begin{document}

\begin{titlepage}
\centering
{\scshape\LARGE Université de Bordeaux \par}
{\scshape\Large Master 1 Informatique \par}
\vspace{3cm}

{\Huge\bfseries Projet de Programmation \par}%%%%
\vspace{0.5cm}
{\Large\itshape Réalité augmentée \par}

\vfill
réalisé par \par
Aghiles \textsc{CHAOUCHI} \par
Nadhir \textsc{HOUARI} \par
Anas \textsc{BELMEKKI} \par
El Mehdi \textsc{KHACHIR} \par \par
Charger de td \par
Pascal \textsc{DESBARATS} \par \par
Client \par
Adrien \textsc{BOUSSICAULT} \par
\vfill

{\large 08 janvier 2017\par}

\end{titlepage}

\section{Besoins fonctionnels}
\subsection{Cas d'une lampe}

\begin{itemize}
  \item Une interface doit s'afficher sur l'écran de la tablette quand l'utilisateur la pointe sur le marqueur.
  \item L'état de la lampe doit être modifiable à partir de la tablette.
  \item L'utilisateur doit pouvoir allumer la lampe grâce à la tablette.
  \item L'utilisateur doit pouvoir éteindre la lampe grâce à la tablette.
  \item L'interface doit disparaitre lorsque l'utilisateur ne pointe plus sa tablette sur le marqueur.
\end{itemize}

\subsection{Cas d'une fenêtre}
\begin{itemize} 
  \item Un ascenseur doit apparaître sur l'écran de la tablette quand l'utilisateur la pointe sur le marqueur.
  \item L'état des volets doit être modifiable à partir de la tablette.
  \item L'utilisateur doit pouvoir faire descendre les volets grâce à la tablette.
  \item L'utilisateur doit pouvoir faire monter les volets grâce à la tablette.
  \item L'interface doit disparaitre lorsque l'utilisateur ne pointe plus sa tablette sur le marqueur.
\end{itemize}

\subsection{Cas d'une porte}
\begin{itemize} 
  \item Un verrou doit apparaître sur l'écran de la tablette quand l'utilisateur la pointe sur le marqueur.
  \item L'état de la porte doit être modifiable à partir de la tablette.
  \item L'utilisateur doit pouvoir verrouiller la porte grâce à la tablette, en cliquant sur le verrou si celle-ci est déverrouiller.
  \item L'utilisateur doit pouvoir déverrouiller la porte grâce à la tablette, en cliquant sur le verrou si celle-ci est verrouiller.
  \item Le verrou doit disparaitre lorsque l'utilisateur ne pointe plus sa tablette sur le marqueur.
\end{itemize}

\subsection{Cas d'un réfrigérateur}
\begin{itemize} 
 \item La façade d'un réfrigérateur doit apparaître sur l'écran de la tablette lorsque l'utilisateur la pointe sur le marqueur.
 \item L'ouverture et la fermeture des deux niveaux (réfrigérateur et congélateur) sont contrôlables à partir de la tablette.
 \item Le réglage de la température des deux niveaux (réfrigérateur et congélateur) est contrôlable à partir de la tablette.
 \item L'utilisateur doit pouvoir ouvrir et fermer l'une des deux portes grâce à la tablette.
 \item L'utilisateur doit pouvoir régler la température des deux niveaux grâce à la tablette.
 \item La façade du réfrigérateur doit disparaitre de l'écran de la tablette lorsque l'utilisateur ne la pointe plus sur le marqueur.
\end{itemize}

\subsection{Cas d'un robinet}
\begin{itemize} 
  \item Un mitigeur doit apparaître sur l'écran de la tablette quand l'utilisateur la pointe sur le marqueur.
  \item L'état du mitigeur doit être modifiable à partir de la tablette.
  \item L'utilisateur doit pouvoir ouvrir et fermer le robinet grâce à la tablette.
  \item L'utilisateur doit pouvoir contrôler le niveau de l'eau froide et de l'eau chaude grâce à la tablette.
  \item Le mitigeur doit disparaitre de l'écran de la tablette lorsque l'utilisateur ne pointe plus sa tablette sur le marqueur.
\end{itemize}

\newpage
\section{Elements bibliographiques}

\begin{itemize}
  \item Pour bien comprendre le concept et l'application concrète de la réalité augmentée, nous avons choisi le livre "UNDERSTANDING AUGMENTED REALITY "\cite{Ref1} qui explique le fonctionnement de la réalité augmentée, et ses différentes applications dans divers domaines.
  \item Afin d'apprendre plus sur les cas d'utilisation de la technologie rasberry Pi, on a eu recours à cet article qui nous a éclairés sur l'utilisation de rasberry pi afin de sécuriser un domicile ainsi que sur la manière d'automatiser le fonctionnement des appareils \'electrom\'enagers \cite{Ref2}.
  \item Afin de tester nos travaux, nous devrons avoir recours à un nano-ordinateur très certainement du type Raspberrypi, pour cela, nous avons inclus la documentation en ligne et officielle de la technologie Raspberrypi \cite{Ref3}.
  \item Nous pouvons aussi utiliser une technologie Arduino à fin de mettre en oeuvre notre travail, pour cela, nous avons inclus la documentation en ligne et officielle de la technologie Arduino \cite{Ref4}. 
\bibliography{bib_bibliographie.bib} %ref au fichier bib
\end{itemize}


\end{document}

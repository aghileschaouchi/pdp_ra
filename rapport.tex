\documentclass[12pt,a4paper]{article}

\usepackage{fullpage}
\usepackage[utf8]{inputenc}
\usepackage{amsfonts}
\usepackage{amssymb}
\usepackage{listings}
\usepackage[french]{babel}
\usepackage[cyr]{aeguill}
\usepackage[hyphens,spaces,obeyspaces]{url}
\usepackage[colorlinks,allcolors=blue]{hyperref}
\usepackage[numbers]{natbib}
\bibliographystyle{plainnat} 

\def\dollar{\$}

\setlength{\parindent}{2em}
\setlength{\parskip}{1em}

\newcommand{\quotes}[1]{``#1''}

\begin{document}

\begin{titlepage}
\centering
{\scshape\LARGE Université de Bordeaux \par}
{\scshape\Large Master 1 Informatique \par}
\vspace{3cm}

{\Huge\bfseries Projet de Programmation \par}%%%%
\vspace{0.5cm}
{\Large\itshape Réalité augmentée \par}

\vfill
réalisé par \par
Aghiles \textsc{CHAOUCHI} \par
Nadhir \textsc{HOUARI} \par
Anas \textsc{BELMEKKI} \par
El Mehdi \textsc{KHACHIR} \par \par
Charger de td \par
Pascal \textsc{DESBARATS} \par \par
Clients \par
Adrien \textsc{BOUSSICAULT} \par
\vfill

{\large 08 janvier 2017\par}

\end{titlepage}

\section{Introduction}

\section{But du document}

\section{But du projet}

L'objectif de ce projet est de permettre à un utilisateur d'interagir avec son environnement à l'aide d'un système d'acquisition et de retransmission vidéo.

\section{Public visé}

\section{Description de la Réalité Augmentée}

\section{Besoins fonctionnels}
\begin{itemize}
 \item Détecter la présence d'un marqueur dans le flux vidéo de la tablette quand l'application est exécutée.
 \item Sélectionner un marqueur en cas où on en a plusieurs qui ont été capturés par la tablette, et ceci par le doigt ou via un laser.
 \item Récupérer un marqueur, l'identifier et l'analyser.
 \item Affecter une fonction à chaque marqueur.
 \item Afficher sur l'écran une interface spécifique à chaque marqueur.
 \item Capturer les manipulations effectuées sur l'interface spécifique à chaque marqueur.
 \item Envoyer les données spécifiques à la manipulation effectuée au serveur qui se doit de changer l'état de l'appareil et ceci via un protocole de communication.
 \item Vérifier que l'information a bien été traiter et que l'appareil à bien changer d'état après la manipulation sur le logiciel (via le protocole de communication).
 \item Signaler sur la tablette le fait que l'appareil ait changé d'état.
 \item Modifier l'interface graphique selon l'état de l'appareil.
\end{itemize}

\section{Besoins non-fonctionnels}
\begin{itemize}
  \item Facilité d'utilisation du logiciel (interface graphique intuitive, temps d'apprentissage minimal).
  \item Fiabilité du logiciel, le logiciel doit comprendre une sûreté d'exécution à fin de ne pas abimer les appareils qu'on se doit de manipuler via le logiciel.
  \item Maintenabilité, robustesse et modifiabilité du logiciel, le logiciel assure le traitement des erreurs via des assertions et des exceptions, ainsi que des tests associés.
  \item Développement sous Android, donc sous langage Java (contrainte).
  \item Utilisation de RasberryPi et d'Arduino comme outils de développements (contrainte).
  \item Extensibilité du logiciel, à partir de besoins fonctionnels généraux on doit pouvoir étendre notre logiciel en ajoutant la gestion d'autres types d'appareils (symboliser par des marqueurs).
  \item Portabilité du logiciel, non seulement en multiplate-forme, mais portable vers d'autres types d'appareils (tablettes, smartphones..).
  \item Performance du logiciel, c'est-à-dire; fluidité du flux vidéo (au moins traitement de 25 images par seconde dans une tablette ou un smartphone Android).
\end{itemize}

\section{Cas d'utilisations}
\subsection{Cas d'une lampe}

\begin{itemize}
  \item Une interface doit s'afficher sur l'écran de la tablette quand l'utilisateur la pointe sur le marqueur.
  \item L'état de la lampe doit être modifiable à partir de la tablette.
  \item L'utilisateur doit pouvoir allumer la lampe grâce à la tablette.
  \item L'utilisateur doit pouvoir éteindre la lampe grâce à la tablette.
  \item L'interface doit disparaitre lorsque l'utilisateur ne pointe plus sa tablette sur le marqueur.
\end{itemize}

\subsection{Cas d'une fenêtre}
\begin{itemize} 
  \item Un ascenseur doit apparaître sur l'écran de la tablette quand l'utilisateur la pointe sur le marqueur.
  \item L'état des volets doit être modifiable à partir de la tablette.
  \item L'utilisateur doit pouvoir faire descendre les volets grâce à la tablette.
  \item L'utilisateur doit pouvoir faire monter les volets grâce à la tablette.
  \item L'interface doit disparaitre lorsque l'utilisateur ne pointe plus sa tablette sur le marqueur.
\end{itemize}

\subsection{Cas d'une porte}
\begin{itemize} 
  \item Un verrou doit apparaître sur l'écran de la tablette quand l'utilisateur la pointe sur le marqueur.
  \item L'état de la porte doit être modifiable à partir de la tablette.
  \item L'utilisateur doit pouvoir verrouiller la porte grâce à la tablette, en cliquant sur le verrou si celle-ci est déverrouiller.
  \item L'utilisateur doit pouvoir déverrouiller la porte grâce à la tablette, en cliquant sur le verrou si celle-ci est verrouiller.
  \item Le verrou doit disparaitre lorsque l'utilisateur ne pointe plus sa tablette sur le marqueur.
\end{itemize}

\subsection{Cas d'un réfrigérateur}
\begin{itemize} 
 \item La façade d'un réfrigérateur doit apparaître sur l'écran de la tablette lorsque l'utilisateur la pointe sur le marqueur.
 \item L'ouverture et la fermeture des deux niveaux (réfrigérateur et congélateur) sont contrôlables à partir de la tablette.
 \item Le réglage de la température des deux niveaux (réfrigérateur et congélateur) est contrôlable à partir de la tablette.
 \item L'utilisateur doit pouvoir ouvrir et fermer l'une des deux portes grâce à la tablette.
 \item L'utilisateur doit pouvoir régler la température des deux niveaux grâce à la tablette.
 \item La façade du réfrigérateur doit disparaitre de l'écran de la tablette lorsque l'utilisateur ne la pointe plus sur le marqueur.
\end{itemize}

\subsection{Cas d'un robinet}
\begin{itemize} 
  \item Un mitigeur doit apparaître sur l'écran de la tablette quand l'utilisateur la pointe sur le marqueur.
  \item L'état du mitigeur doit être modifiable à partir de la tablette.
  \item L'utilisateur doit pouvoir ouvrir et fermer le robinet grâce à la tablette.
  \item L'utilisateur doit pouvoir contrôler le niveau de l'eau froide et de l'eau chaude grâce à la tablette.
  \item Le mitigeur doit disparaitre de l'écran de la tablette lorsque l'utilisateur ne pointe plus sa tablette sur le marqueur.
\end{itemize}

\subsection{Cas d'un ventilateur}
\begin{itemize} 
  \item Une interface doit apparaître sur l'écran de la tablette quand l'utilisateur la pointe sur le marqueur.
  \item l'état du ventilateur doit être modifiable à partir de la tablette. 
  \item L utilisateur doit pouvoir démarrer et éteindre le ventilateur. 
  \item L utilisateur doit pouvoir régler le degré de la ventilation.
  \item L'interface doit disparaitre de l'écran de la tablette lorsque l'utilisateur ne pointe plus sa tablette sur le marqueur.
\end{itemize}

\subsection{Cas d'un téléviseur}
\begin{itemize} 
  \item Une interface doit apparaître sur l'écran de la tablette quand l'utilisateur la pointe sur le marqueur.
  \item L utilisateur doit pouvoir changer les chaines.
  \item L utilisateur doit pouvoir modifier le son.
  \item L utilisateur doit pouvoir allumer et éteindre la télévision.
  \item L'interface doit disparaitre de l'écran de la tablette lorsque l'utilisateur ne pointe plus sa tablette sur le marqueur.
\end{itemize}

\newpage
\section{Elements bibliographiques}

\begin{itemize}
  \item Pour bien comprendre le concept et l'application concrète de la réalité augmentée, nous avons choisi le livre "UNDERSTANDING AUGMENTED REALITY "\cite{Ref1} qui explique le fonctionnement de la réalité augmentée, et ses différentes applications dans divers domaines.
  \item Afin d'apprendre plus sur les cas d'utilisation de la technologie rasberry Pi, on a eu recours à cet article qui nous a éclairés sur l'utilisation de rasberry pi afin de sécuriser un domicile ainsi que sur la manière d'automatiser le fonctionnement des appareils \'electrom\'enagers \cite{Ref2}.
  \item Afin de tester nos travaux, nous devrons avoir recours à un nano-ordinateur très certainement du type Raspberrypi, pour cela, nous avons inclus la documentation en ligne et officielle de la technologie Raspberrypi \cite{Ref3}.
  \item Nous pouvons aussi utiliser une technologie Arduino à fin de mettre en oeuvre notre travail, pour cela, nous avons inclus la documentation en ligne et officielle de la technologie Arduino \cite{Ref4}.
  \item Le code QR est un type de code-barres en deux dimensions constitué de modules noirs disposés dans un carré à fond blanc. Définition et méthodes de détection d'un QR code \cite{Ref5}.
  \item Pour pouvoir extraire une image dans un flux vidéo \cite{Ref6}.
  \item La domotique le grand domaine de notre sujet "réalité augmentée" \cite{Ref7}.
  \item Dans le dessin de piloter les objets nous aurons besoin de programmer un circuit arduino \cite{Ref8}.
  \item Définition et déclenchement d'un marqueur ArUco \cite{Ref9}.
    \item Comparaison entre les protocoles domotiques selon leur efficacité, économie et  simplicité de mise en oeuvre
      \cite{Ref10}.
      \item Liste de quelques applications domotique disponible sur Android
        \cite{Ref11}.
        \item Définitions et concept de la domotique 
          \cite{Ref12}.
           \item L'ouvrage "Arduino pour la domotique" permet de s'initier à la domotique avec Arduino
             \cite{Ref13}.
              \item "Raspberry Pi et l'ESP 8266 pour la domotique" est un ouvrage qui détaille l'utilisation de Raspberry Pi et l'ESP 8266 pour réaliser des applications de la domotique
  \cite{Ref14}.
    \bibliography{rapport.bib} %ref au fichier bib
\end{itemize}

\end{document}
